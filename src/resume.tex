\documentclass[10pt,a4paper,sans]{moderncv}

% moderncv themes
\usepackage[french]{babel}
\moderncvstyle{custom}
\moderncvcolor{purple}
\nopagenumbers{}

% character encoding
\usepackage[utf8]{inputenc}

% adjust the page margins
\usepackage[scale=0.82, top=2cm, bottom=2cm]{geometry}

% personal data
\name{Kevin}{Houdebert}
\title{Ingénieur système et infrastructure}
\address{1 rue Robert Schumann}{Le Perreux sur Marne - 94170}
\phone[mobile]{06 11 96 56 56}
\email{kevin@qwazerty.eu}
\homepage{github.com/qwazerty}
\photo[64pt][0.4pt]{id}

%------------------------------------------------------------------------------
%            content
%------------------------------------------------------------------------------
\begin{document}
%-----       resume       -----------------------------------------------------
\makecvtitle

\vspace{-3\baselineskip}

\section{Formation}
\cventry{sept 2010 -- sept 2015 (attendu)}
    {École Pour l'Informatique et les Techniques Avancées}
    {EPITA}{Kremlin-Bicêtre}{}
    {Étudiant en 5\up{ème} année, spécialité temps réel et embarqué}
\cventry{janv 2012 -- juin 2012}
    {Semestre universitaire}{Brock University}{Ontario, Canada}{}
    {Procedural Programming, Digital Electronics, Abstract Linear Algebra}
\cventry{2010}{BAC Scientifique}{Lycée Claude Bernard}
    {Paris}{\textit{Mention Assez Bien}}{}

\section{Projets réalisés}
\cventry{2014}
    {C -- Éxecution de code dans une VM en utilisant l'API kernel de KVM}
    {MyKVM}{2 mois}{}{}
\cventry{2014}
    {C++ -- Simulation du comportement du Pipeline MIPS}
    {Pipeline MIPS}{2 mois}{}{}
\cventry{2012}
    {C -- Création d'un shell similaire à Bash, norme POSIX (Équipe de 4)}
    {42sh}{3 semaines}{}{}

\section{Expérience}
\cventry{mars 2015 -- sept 2015}
    {Stage ingénieur système et infrastructure}{Smile Open Source Solutions}{}{}
    {Étude déploiement continu avec Jenkins, Ansible, et OpenStack (Linux)}
\cventry{janv 2014 -- déc 2014}
    {Administrateur système et réseau Linux}{YAKA - ACU}{}{}
    {Maintenance de services pour les élèves d'EPITA (1000 étudiants) (Linux)}
\cventry{août 2013 -- déc 2013}
    {Stage développeur d'applications distribués}{Qarnot Computing}{}{}
    {Implémentation d'une architecture client/serveur sur cartes embarquées (Linux, Docker, Python, C)}
\cventry{sept 2012 -- juin 2013}
    {Assistant C\#/Caml}{ACDC}{}{}
    {Organisation de sujets, TP, et correction des élèves d'EPITA en première année (Linux, C\#, Caml)}
\cventry{juin 2012 -- août 2012}
    {Stage développeur Web}{Soixante Circuits}{}{}
    {Développement de sites Web en PHP/JavaScript (jQuery) avec CMS WordPress}
\cventry{juin 2011 -- juill 2011}
    {Stage saisonnier}{THALES Air Systems}{}{}
    {Mise à jour de base de données documentaire}
\cventry{2010 -- 2011}
    {Gestion d'un site associatif de défense des droits}
    {Association}{}{}
    {Amélioration graphique du site web et mise en forme d'articles régulièrement}

\section{Compétences}
\cvdoubleitem{Langages}{sh, Python, C, C++, C\#, Java, \newline{}ASM (SPARC, x86\_64)}
    {Systemes}{GNU/Linux (Arch Linux/Debian), \newline{}FreeBSD, Windows}
\cvdoubleitem{Outils}{Git, Puppet, Ansible, Docker, Libvirt}{}{}
\cvdoubleitem{Anglais}{Courant (TOEIC: 905/990)}{Français}{Langue maternelle}
\cvitem{Polonais}{Parlé}

\section{Centres d'intêret}
Programmation, musique (métal, électro), jeux-vidéos

\clearpage
%-----       letter       ---------------------------------------------------------
% =============================================================================
% #                  UNCOMMENT FOLLOWING FOR THE LETTER                       #
% =============================================================================
%% recipient data
%\recipient{RECIPIENT}{}
%\date{DATE}
%\opening{OPENING}
%\closing{CLOSING}
%%\enclosure[Attached]{curriculum vit\ae{}}
%\makelettertitle
%\linespread{2}\selectfont
%
%FIXME
%
%\makeletterclosing
\end{document}


%% end of file `template.tex'.
